\documentclass[
a4paper,
12pt
]{scrartcl}


\usepackage{hyperref}
\usepackage{url}

\usepackage[scaled=0.9]{helvet}
\usepackage{mathpazo}
\usepackage{enumitem}
\selectfont
\linespread{1.25}


\usepackage[english]{babel}
\usepackage[utf8]{inputenc}
\usepackage[T1]{fontenc}
\usepackage{graphicx}
\usepackage[font=small, labelfont=bf]{caption}
\usepackage[list=true, font=small, labelfont=bf, 
labelformat=brace, position=top]{subcaption}

\usepackage{amstext}

\usepackage{courier}
\usepackage{caption}

\title{Interactive Planning}
\subtitle{Russel L. Ackoff}
\author{\\\vspace{-0.4em}{\normalsize Daniel Werner} \\ {\normalsize Universität Leipzig}}
\date{{\normalsize July 6, 2021}}


\begin{document}
	\maketitle
	\section{Introduction}
	
	Interactive Planning is a methodology derived by Russel L. Ackoff, in his  book \glqq Creating the Corporate Future: Plan or Be Planned for\grqq  \cite{ackoff:2001}. It is build upon the basic concept that the future of a company depends on what actions and events it realizes in the present, aiming towards an ideal  future. Interactive Planning (\textit{IP}), and trough its execution their planner, wants to design a desirable present which it then tries to approximate as good as possible \cite{ackoff:2001}. The approach clearly sees the interdependence of problems and incorporates this into its planning.
	
	Ackoff based this idea on the ideal of his so called „Interactivist“ , who wants to actively shape the future of the organization. The „Interactivist“ accepts that the future cannot be predicted  and planned for in an all-encompassing way.    
   
   \section{The three principles of Interactive Planning}
     Three principles are constantly influencing Interactive Planning \cite{giannaris:2011}:
     \begin{itemize}
     	\item The participative principle: \\Involvement of as many different stakeholders in the process as possible. All planning members should learn to understand the organization as a whole.
     	
     	\item The principle of continuity:\\Since \textit{IP} is not based on future predictions, the plans that have been developed must always be monitored, evaluated and possibly modified.
     	
     	\item The holistic principle:\\Relevance of simultaneous and interdependent planning across all levels of the organization. Coordination (all parts across the same organization level) and Integration (one part of the system across all levels of organization) always have to be kept in mind while planning.
     \end{itemize}
     
     
     \section{The Phases of Interactive Planning}
	
	Interactive Planning consists of two parts, Idealization and Realization, which each consist of planning phases. Idealization has two phases, Formulating the Mess and Ends Planning, whereas Realization has the remaining four phases, Means Planning, Resource Planning, Design of Implementation and Design of Controls.
	
	\subsection{Idealization}
	\paragraph{Formulating the Mess} 
	
	This first phase can be viewed as a situational analysis. The term "mess" describes the multiple, interacting threats the organization will face in the future (unless it changes). The goal is to find the reasons for the organization's potential decline if it does not adapt to its environment. It consists of four sub-activities: 
	
	\begin{itemize}
		\item System Analysis:\\
		A detailed description of how the system (organization + environment) functions and operates,  the organizational structure, policies, strategies, and practices.
		
			\item Obstruction Analysis:\\
			Identification of attributes and characteristics that impede the organization's development and identification of conflicts within the entire system itself.
			
			\item Reference Projections:\\
			Extrapolation of current organizational data and performance characteristics into the future, assuming no changes are happening in the organization or its environment.
			
			\item Reference Scenario:\\
			A detailed, procative, and possibly even shocking description of how the organization would end itself (self-destruct) on its current path, assuming the previous analysis is to be true \cite{lumbo:2007}. Systematically, it is the synthesis of the previous three steps that yields the "mess", the disorder in which the organization finds itself. 
			The objectives of this reference scenario are to highlight implications of current behavior and draw attention to relevant problems. Also its aim is to motivate all stakeholders to change and improve the organization.
	\end{itemize}
	
	
		\paragraph{Ends Planning} 
	
	This second phase is probably the most complex and important one. The planners define what the organization would like to be at the present time. It then aims at identifying the discrepancies between the developed Reference Scenario and the designed desired present. The "Ends" are then the goals to be achieved, the formulation of the ideal of the organization. 
	For this purpose, Ackoff describes the methodology of "Idealized Design".
	
	\subparagraph{Idealized Design}
	The basic assumption of the approach is, that the organization to be planned was destroyed last night, but its environment in which it was embedded remains intact. On this base the planners should design an organization to replace the current (destroyed) one \textbf{right now}.
	Every possible organization is conceivable, except for two constrains and one prerequisite \cite{ackoff:2001}:
	
	\begin{itemize}
		\item Technological Feasibility: The design should only use technologies that are usable at the current time.
		
		\item Operational Viability: The system, should it begin to exist, is able to survive in its current environment.
		
		\item Learning and Adaptation: The organization should have the ability to continuously adapt to internal and external changes that could potentially affect it. This also implies  the requirement for adaptability of internal and external stakeholders.
	\end{itemize}

Idealized Design consists of three phases:

\begin{enumerate}
	\item Formulation of a mission statement,
	\item Specification of the characteristics that the organization to be designed should possess,
	\item Design of an organization with these characteristics.
\end{enumerate}

\noindent The goal of "Idealized Design" is \textbf{not} an ideal organization, but a possible, best result at the time of development.

		\begin{quote}\itshape{\glqq
		It is neither perfect nor utopian. The design produced should be that of the best ideal-seeking system of which its designers can currently conceive. (They may, and probably will, be able to conceive of a better one later.)
		\grqq } \cite{ackoff:2001}
\end{quote}	



	
    \subsection{Realization}
		
	\paragraph{Means Planning} 
	
	This phase is the first of the realization step. It aims at the development of means/opportunities to close or at least reduce these discrepancies or "gaps" identified in Ends Planning.  Therefore, it can be seen as the correlation of the reference scenario and idealized design \cite{lumbo:2007}. Here, the planners elaborate and select courses of action, projects, programs, and new policies that drive the organization closer to the ideal.
	The planned ideal present should be approximated as best as possible for the near future.
	
	Problems in the Reference Scenario can be handled by either "resolving, solving or absolving" \cite{ackoff:1981}. Absolving (justifying) should rarely be chosen and should be done only under certain circumstances.
	It is better to find a solution to the problem (solve) or at least to eliminate it (resolve).  For each problem, several alternative solutions can be discussed, which are then prioritized and selected by means of questioning, experiments, models or simulations \cite{ackoff:1981}.
	
	
	\paragraph{Resource  Planning} 
	
	The means and possibilities developed are now considered in the context of economic and business aspects, specifically under the following questions:
	\begin{itemize}
		\item What  and how many resources are needed to implement the Means? Where are they needed?
		\item When will the resources be needed and how much will be available?
		\item What should happen in the event of a shortage or surplus of resources?
	\end{itemize}

    Ackoff identifies five relevant categories of resources for planning: inputs (e.g. materials, supplies, energy and services), facilities and equipment, personnel, money and data (information, knowledge, understanding and wisdom).\cite{ackoff:2001} 
    
    
		
	\paragraph{Design of Implementation} 
	
	This phase aims at planning and executing the previously developed means (in the context of resources). Decisions made in the previous phases are translated into a set of instructions and schedules. Those responsible for planning should fully and holistically coordinate this process, and be available as contact persons. This phase was briefly and simply summarized by Ackoff as \glqq
	determining who is to do what, when and where.\grqq \cite{ackoff:2001}
	
	
   \paragraph{Design of Control} 
   
	This final phase runs in parallel with the previous one as a control and monitoring instance. Criteria are identified and selected that allow evaluation of the success of planning decisions. Using these developed metrics, the instructions and flowcharts are then to be monitored, checked for effectiveness, and adjusted if necessary in the event of errors.
	
	
    \subsection{Execution of Interactive Planning}
	
	According to Ackoff, these six phases should be initialized in this order, but need not be explicitly performed in the presented order. Due to their strong dependence on each other, they often take place simultaneously and interactively. Interactive planning is therefore also continuous planning, in which no phase is ever completed.	
	
		\begin{quote}\itshape{\glqq
		All outputs are subject to subsequent revision. Plans are treated as, at best, still photographs taken from a motion picture.\grqq } \cite{ackoff:2001}
\end{quote}	
	
	
	
	
	\newpage
	
	\bibliographystyle{acm}
	\begin{thebibliography}{xxx}
		
		\bibitem{ackoff:2001}  Ackoff, Russell L.
		\newblock {\em A Brief Guide to Interactive Planning and Idealized Design}.  
		\newblock IDA Publishing, May 31, 2001.
		\newblock \url{https://www.ida.liu.se/~steho87/und/htdd01/AckoffGuidetoIdealizedRedesign.pdf},
		\newblock Accessed 22 June, 2021.
		
		\bibitem{ackoff:1981}  Ackoff, Russell L.
		\newblock {\em Creating the Corporate Future: Plan or Be Planned for}.
		\newblock New York:	Wiley, 1981.
		
		\bibitem{giannaris:2011}  Giannaris, Pericles
		\newblock {\em Implementation Of Interactive Planning}.
		\newblock Master of Science in Organizational Dynamics Theses, September 09, 2011.
		\newblock \url{https://repository.upenn.edu/cgi/viewcontent.cgi?article=1045&context=od_theses_msod},
		\newblock Accessed 23 June, 2021.
		
		\bibitem{lumbo:2007}  Lumbo, Donna
		\newblock {\em  Applications of Interactive Planning Methodology}.
		\newblock Master of Science in Organizational Dynamics Theses, April 03, 2007.
		\newblock \url{https://repository.upenn.edu/cgi/viewcontent.cgi?article=1002&context=od_theses_msod},
		\newblock Accessed 23 June, 2021.
	
	\end{thebibliography}

	
\end{document}

  \cleardoublepage