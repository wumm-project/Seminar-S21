\documentclass[a4paper,11pt]{article}
\usepackage{a4wide,url,enumitem}
\usepackage[english]{babel}
\usepackage[utf8]{inputenc}

\parindent0pt
\parskip3pt
\setlist{noitemsep}

\title{Anton Kozhemyako. Contradictory Business Processes and Schematization}
\author{Lukas Heink, Leipzig University}
\date{July 13, 2021}


\begin{document}
\maketitle

\section{Introduction}

In his thesis “Special Features of the Use of TRIZ for Solving
Organisational-Managerial Tasks (OMT): Schematization of an Inventive
Situation and Work with Contradictions” \cite{K1} Anton Kozhemyako proposes a
way to use schematization on business tasks to TRIZ tools which can be
successfully applied. Further he develops a method for determining the
operational zone in OMTs.

\section{Organisation and Management}

According to Kozhemyako \cite{K1} there are three forms of activities that
build the basis of management: 
\begin{itemize}
\item \textbf{Organisation} is the process of forming supersystems and/or
  subsystems of various level in business systems: associations,
  organisations, departments and workplaces. Organisation is the formation of
  the structure, i.e. of the elements and their interconnections.
\item \textbf{Leadership} is the setting of tasks to performers and monitoring
  their implementation.
\item \textbf{Management} is a change in the activities of performers, i.e.
  when the structure is organised and all tasks are distributed (including
  tasks for feedback), but the efficiency of the performers is not
  satisfactory.  The Manager tries to change their activity in the direction
  of improvement, that is, he/she begins to manage their activity.
\end{itemize}
All of these are activities in business systems and can be called
“organisational-managerial” tasks or OMTs for short.

Most of such problems do not cause difficulties to managers since they
encounter similar situations regularly. However, some OMTs cannot be solved in
the usual way.  For this reason, many attempts have been made to use TRIZ
tools to solve OMTs but to analyse the OMT it turned out to be difficult or
unreliable since most of the TRIZ methods are ill equipped to incorporate
human elements \cite{K1}. When solving OMT, it is impossible to consider
people in organised social systems as "objects", since they are essential (and
often the most important) elements of the system.

\section{Schematization}

Schematization is a method developed within the Moscow Methodological Circle
lead by G.P. Schedrovitsky as a means to  solve problematic situations in the
field of organization and management \cite{K1}. 

Schematization tries to look at the business system from a bird’s eye view.
The use of schematization prevents the narrowing of the task during the
analysis stage \cite{K3}.

Components of schematization: \cite{K3}
\begin{itemize}
\item System Framework or model of a working system (MWS). This is the system
  to be analysed.
\item Elements. There are two types of elements: Objects and Subjects.
\item Levels. The levels describe which element is managing and which element
  is managed. The element of the higher level is the managing one.
\item Connections. There are three types of connections:
  \begin{itemize}
  \item A direct line – A relation. There is a connection between two elements
    but it is of no interest.
  \item A one-directional arrow – A function. A function is defined similar to
    TRIZ.
  \item A two-directional arrow – A process. A process is the development of a
    phenomenon in time.
  \end{itemize}
\item Generalized objects. A generalized object is a shell that serves a role
  within a system. It specifies the requirement of that element.
\item Filling (content). A filling or content of a generalized object
  specifies the requirement of a specific entity. For example, a person with
  relevant competences or a computer program with special characteristics.
\end{itemize}
Using a generalized object without a filling result in a repetitive solution
that can be used in a variety of situations and it can be easier to anticipate
outcomes.

Using a filling on the other hand allows for the usage of its characteristics
to get a more specific (taylored) solution for the problem but risks not being
reusable.

Schematization should be considered only as a tool for the primary analysis of
a business system and should be combined with the analytical tools of TRIZ
\cite{K1}.

\section{Inherent Contradictions of Goals Regarding the\\ Remodelling of
  Business Processes} 

When remodelling business processes the main technical contradiction has to be
found. The technical contradiction is a pair of two opposing goals both
demanding the same resource. This pair is called the operational zone and is
the location of the conflict.

Within the operational zone you can find a tool, which is the object that
performs a negative impact, and a product which receives the impact as well as
the environment surrounding this conflicting pair.

Using those it is possible to find the resources used by these elements and
prioritise them if necessary. Resources can be prioritised by e.g. the
available quantity, how useful or harmful it is, how high the cost of the
resource is or to which element it correlates.

Each resource should be used in the most optimal way possible in order to get
the best solution~\cite{K1}. 

\raggedright
\begin{thebibliography}{xxx}
\bibitem{K1} Anton Kozhemyako (2019).  \emph{Special Features of the Use of
  TRIZ for Solving Organisational-Managerial Tasks (OMT): Schematization of an
  Inventive Situation and Work with Contradictions}.
  \url{https://bit.ly/3z5Mdx1}. Accessed 7 July, 2021
\bibitem{K2} Anton Kozhemyako (2020). Comparison of three different TRIZ tools
  for a business problem analysis considering the example of the mentoring
  problem in a sales department. TRIZ Review. Vol 2, \#1, April 2020,
  pp. 81--99.
\bibitem{K3} Valeri Souchkov (2019). MATRIZ Scientific Adviser Review to
  \cite{K1}.  \url{https://matriz.org/kozhemyako/}. Accessed 10 July 2021
\bibitem{K4} Ilya Iljin (2019). MATRIZ Official Critique to the TRIZ Master
  Thesis \cite{K1}.  \url{https://matriz.org/kozhemyako/}. Accessed 10 July
  2021
\bibitem{K5} Anton Kozhemyako (2020). Getting started on a business problem.
  Talk at the Business TRIZ Online 2021.
  \url{https://bit.ly/3ARhFAJ}. Accessed 12 July 2021
\end{thebibliography}
\end{document}
