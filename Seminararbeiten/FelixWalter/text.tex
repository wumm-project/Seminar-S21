\documentclass[a4paper,12pt]{article}
\usepackage{hyperref,graphicx,a4wide,enumitem}
\usepackage[english]{babel}
\usepackage[utf8]{inputenc}

\parindent0pt
\parskip3pt

\renewcommand\refname{References and Notes}

\title{Criticism of Henry Mintzberg's Theses on Management}

\author{Felix Walter}

\date{October 31, 2021}
\begin{document} 
\maketitle 

\begin{abstract}
This paper was written in the context of the seminar Complex Systems and
Co-Operative Action and serves as a critical comparison of arguments in
relation to Henry Mintzberg's theses about management and the social processes
in the form of organizations.
\end{abstract}
\begin{quote}
  “I argue for a return to balance, for intuition to be allowed alongside
  analysis and recognized as a necessary process in organizations.” - Henry
  Mintzberg \cite{Mintzberg}
\end{quote}
\thispagestyle{empty}

\tableofcontents
\newpage

\section{Introduction}

For the consideration of a system-theoretical comparison of management
theories an analysis from different points of view is to be made, which makes
it possible to point out, apart from socially available, individual knowledge
of management, a structured, practical procedure, which makes an
institutionalized application comprehensible. Furthermore, the tactical,
momentary behavior at the management level is abstracted so that it can be
conceptualized a conceptual generalization regarding the structures of an
organization and its management and further meet a scientific reflection that
summarizes the methodology of management and generates tools of analysis.

In this paper, Henry Mintzberg's theses on management and organizations will
be critically examined with the help of concepts from management theory and
technical systems. The aim is to question Mintzberg's analogies and concepts
from a real-world implementation and to compare his view with concepts of
systems theory. Special focus is put on the development of strategies in
relation to management processes and how they are applied in an organization
according to Mintzberg. The characteristics of different forms of
organizations are compared with the concepts of a socio-technical system and
put into context. Mintzberg's theses on social structural change through
organization will finally be subjected to a critical examination.

Henry Mintzberg (*02.09.1939), a professor of economics and management at
McGill University in Canada, refers in his book "Mintzberg on Management" to
prevailing explanations of how an organization can successfully achieve its
goals through appropriate management by analyzing the subject of the
organizational process, the manager, in its form of execution and highlighting
society as an organizational pattern. The goal of his consideration is to
formally categorize the activities that constitute managers in order to
anticipate assumptions about successful management and, finally, to define
strategy development as a craft process. An emergent approach by focusing on
the characteristics of managers is postulated by him to infer the success of
the overall organization from the capabilities of the manager. This approach
is discussed in more detail below.

\section{Manager}
To study the characteristics of the manager, Mintzberg looks at the everyday
actions of people in management positions in different industries. He compares
various assumptions and relates them to his observations. He begins by stating
the belief that managers must be conscious, systematic planners in order to
run organizations successfully. The extent to which conscious and systematic
activity is designed is not further defined, nor is it explained why these
assumptions should hold. Through his observations, Mintzberg concludes that a
manager's actions are brief, varied, and discontinuous, and beyond that, there
are no planning activities that are manifested through an external
action. These assumptions are substantiated by Mintzberg's research in "The
Nature of Managerial Work", Robert H. Guest "Of Time and the Foreman",
\cite{guest} and Rosemary Stewart "Managers and their Jobs" \cite{stewart}.
This is immediately followed by the criticism of the "classical literature"
that "managing does not produce reflective planners" \cite[p.25]{Mintzberg}.
Mintzberg does not give any examples or details of what "classical literature"
is supposed to be, let alone what exactly it says. According to Mintzberg,
planning or goal setting cannot be actions separate from the work
process. "Managers seem to meet planning needs implicitly in the context of
daily actions, not some kind of abstract process [...]"
\cite[p.25]{Mintzberg}.  Following this, he negates the statement that
managers do not have regular duties to perform by highlighting research in
\cite{Mintzberg2} and Robert T. Davis ``Performance and Development of Field
Sales Managers'' \cite{davis}, which purports to prove that managers perform a
number of regular duties such as ritual, ceremony, and negotiation, as well as
processing soft information in their daily work. No mention is made of how
often managers must perform these "duties," what planning processes might be
set in motion by the "duties," or the extent to which they measurably
contribute to the success of the organization. Next, the assumption that
"managers in top positions [...] need aggregated information [that] is best
provided by a management information system" is questioned, in which Mintzberg
refers to the researches Stewart "Managers and Their Jobs" \cite{stewart2} and
Tom Bums, "The Directions of Activity and Communication in a Departmental
Executive Group" \cite{burns}, in which, according to Mintzberg, it was found
that "managers [...] very much prefer the linguistic medium - i.e. telephoning
and face-to-face meetings" \cite[p. 26]{Mintzberg}. The research mentions that
"managers [spend] an average of 66 to 80 percent of their time on oral
communication. The reason for this is given in the "early warning function" of
so-called "soft information". "Today's gossip can be tomorrow's reality."
\cite[p. 27]{Mintzberg}. It is important to note that the external sources
used are all from the 1950s. This comment is later picked up by Ansoff
\cite{ansoff}.  Neither is this statement supported by facts, nor are figures
given as to how far "gossip" actually occurs in reality. Thus, Mintzberg
further claims that "managers [apparently do not] document [...] what they
hear. Therefore, the strategic database of the organization lies less in the
computers than in the heads of its managers." \cite[p. 27]{Mintzberg}. Why
Mintzberg infers low documentation of planning processes from increased oral
communication is not apparent. The connection between implicit planning by the
subject and lack of explicit presentation of decision patterns is not
explained by any evidence. Already here it becomes apparent that Mintzberg,
starting from the manager's form of description, concludes to a prescription
for successful management, which is not carried out argumentatively, but
elevated to a law by assumptions alone. Later, this logical impurity will be
described more clearly. Last, the assumption that "management [...] [is] a
science and a profession, or [...] [can] quickly develop into one"
{}\cite[p.28]{Mintzberg} shall be refuted, since "managers' programs [...]
[remain] hidden deep in their minds." \cite[p. 28]{Mintzberg}. It is more
closely indicated what Mintzberg means by "deep in the mind" by introducing
the terms "judgment" and "intuition" but not explaining them in any way. Only
the argument of "oral communication" again finds a way into the description:
"The manager is overloaded with obligations: but he cannot simply delegate his
task. Therefore he has to overwork himself and is forced to do a lot of things
superficially. Brevity, fragmentation, and oral communication characterize his
work." \cite[p. 28]{Mintzberg}. Neither evidence of commitment overload nor the
compulsion to overwork and subsequent superficiality is presented. At no time
is it apparent why Mintzberg makes these statements.

\subsection{Roles of the Manager}

He defines the manager's characteristics in terms of several roles a manager
\cite{Mintzberg} assumes, conditioned by his formal authority and status over
the organization. He has responsibility over decisions as well as over
employees and subgroups. Information is gathered and shared by the
manager. The roles are divided into three areas, which he calls interpersonal,
informational and decision-making roles, which are elaborated below.

The interpersonal roles are subdivided into representative figure, managerial
figure, and contact figure. As a representative figure, the manager must
attend ceremonies and conduct routine affairs. There is hardly any
communication that influences important decisions of the
organization. Presence in public is the decisive task here. As a leader, the
manager carries formal leadership over the organization and hires and trains
employees. The potential power enables him to make important decisions and to
instruct and motivate employees for the tasks. As a contact person, important
information is gained, which is secured and expanded by building an external
information system with influential contacts. Communication here is mostly
informal, private and verbal.

Information roles are defined as monitor, distributor, and spokesperson. As
monitor, the manager scans the environment for information and obtains it
through the network of contacts he has built. Gossip, rumors, and speculation
are part of the information-gathering process, which usually occurs in verbal
form. As a distributor, the information gained is passed on to employees who
otherwise would not have access to it or would find it difficult to
access. Going further, as a spokesperson, the manager has the task of making
speeches to external people and advocating for the needs of the
organization. He must inform and satisfy influential people who control the
organizational unit. Directors or shareholders thus receive information about
the financial condition of the organization.

Mintzberg highlights the decision-oriented roles in the four roles of
entrepreneur, crisis manager, resource allocator, and negotiator. As
entrepreneur, the manager initiates new development projects and controls and
delegates them. The department belonging to him should improve and adapt to
the changing environmental conditions. As a crisis manager, he responds to
external constraints that affect the organization. Constraints include
circumstances such as strikes, bankruptcies, or delivery problems. As a
resource allocator, the manager decides who gets what in the organization and
to what extent. Decisions are coordinated so that there are no
discontinuities. Last, as a negotiator, the manager negotiates access to
resources and important information with influential people.

The listed roles of the manager form a gestalt that, according to Mintzberg,
cannot be divided or considered separately. Every manager must pay equal
attention to all roles in order to create the necessary conditions for
successful management. It is the integration of all components that makes a
person a manager. The roles are hierarchized starting from the formal
authority, to the interpersonal, information roles and finally
decision-oriented roles in this order, thus placing the existence condition on
the respective previous roles. This gradation is not explained, nor is
evidence presented for this assumption. In the following explanations these
roles are also not mentioned again at any time or put into relation with the
assertions to the strategy development, organization or society. What function
the presentation of the roles fulfills or why Mintzberg introduces them does
not find any logical justification in his description, nor any practical
application. The form of representation is limited solely to the subjective
view of the manager and is not related to a collective management
process. This fact is later related to the definition of strategy development.


\section{Mintzberg's Model of Strategy Formation}

\subsection{The emergent strategy}

According to Mintzberg, a strategy is "plans for the future and patterns from
the past" \cite[p. 41]{Mintzberg}. The pursuit of this strategy reflects the
practice of a "realized strategy" \cite[p. 41]{Mintzberg}. The development of
a strategy is described as an emergent process that relies solely on
involvement and a sense of familiarity. Mintzberg compares strategy
development to an artistic process of creation based solely on long experience
and involvement. His assumption is therefore that organizations must pursue
emergent strategy development based solely on successful experience and
experimentation. Similar to the artist, the manager surrenders to "calculated
chaos" \cite[p. 40]{Mintzberg}. Strategy development, according to Mintzberg,
generates a logical plan through past patterns of experience. In this process,
strategies are not meant to be made explicit, but to dwell in the "minds of
managers." Moreover, he asserts that strategies should not be formulated in
both unpredictable and predictable environmental conditions. Managers should
not make assumptions about strategy apart from their experience. Two
exceptions are: The organization is newly initiated and must make initial
conceptualizations, or it is coming out of a period of transition into a
stable situation. Mintzberg exacerbates this assumption by saying that "the
long-held image of planning in the literature distorts these processes [of
  strategy development] and [...] thus misleads those organizations that rely
unreservedly on planning." \cite[p. 42]{Mintzberg}. Developing a strategy is a
process that, apart from conscious planning, refers to stability, discovering
discontinuities, knowing the industry, pattern recognition, and the interplay
between change and continuity. Using the term "strategic learning," Mintzberg
points out that "a purely planned strategy [...] precludes [learning] once the
strategy is formulated. The "emergent strategy described earlier reinforces
it" \cite[p. 45]{Mintzberg}. In doing so, he excludes the absolute forms:
"None of the approaches is very useful if pursued in the extreme. Learning
must be coupled to control." \cite[p. 45]{Mintzberg}. Further, Mintzberg
refers to the different strategy approaches in the analogy of the left and
right hemispheres of the brain to represent the opposite forms of rational,
logical and intuitive, artistic strategy development. Accordingly, the left
brain functions in linear patterns and is fixated on logical conclusions. On
the other hand, in the right hemisphere, influences are generally processed
simultaneously and subconsciously. Especially the perception and processing of
images and sensual influences is attributed to the right side. Mintzberg
states in this comparison that managing should be carried out to a high degree
starting from the right brain hemisphere. For example, the recognition of
facial features of other persons, the correct interpretation of voice tones or
gestures or the processing of "soft information" is crucial for the management
process. Preferably, then, there is a relational, simultaneous aggregation of
information that takes place by the manager in the subject. According to
Mintzberg, "hearsay," gossip, hallway conversations, or the like are a
decisive factor for management success. He speaks of an increased synthesis
process in which rational analysis plays a minor role. As a result, Mintzberg
says, "I therefore hypothesize that the important processes of managing
organizations depend in large part on right-brain skills."
\cite[p. 63]{Mintzberg}. Planning, as mentioned above, should take place only
under stable conditions and, complementarily, only when innovative strategy
pursuit is not necessary. Communication of results to top management should
always be verbal. In summary, Mintzberg gives the work instruction: "Plan with
the left, manage with the right" \cite[p. 57]{Mintzberg}. What is clear in his
reflection is that he is particularly critical of the teaching of practices of
management which, according to him, "basically [have] led modern management
schools to worship the left hemisphere." \cite[p. 68]{Mintzberg}. Mintzberg
does not provide any further evidence here outside of his aforementioned
observations showing the failure of the "right hemisphere" to emerge, nor what
he means by stable environmental conditions and innovative strategy measures.

"Some of the best-known management schools have basically become closed
systems in which professors with little interest in organizational reality
teach inexperienced students mathematical, economic, and psychological
theories as ends in themselves. In these management schools, management is
given only a small place. Our schools need a new balance, namely the best
possible balance between analysis and intuition." \cite[p. 68]{Mintzberg}.

Finally, Mintzberg refers to the "insights about consciousness"
\cite[p. 68]{Mintzberg} being suppressed by "artificial mystifying behavior".
Neither the concept of "consciousness", nor the "proper balance between
analysis and intuition" is elaborated by him. The implementation of his
hypothesis in a practical context does not receive any shape and can be
understood only as a criticism of the management schools, which is evident
only in the comparison of the concepts of analysis and intuition.

\subsection{Hemispheres of the brain}

The comparison of left and right brain hemispheres and their functions has
already been considered by Julian Jaynes in his theory of the bicameral mind
\cite{Jaynes}. Jaynes hypothesizes that the emergence of language is a
necessity for consciousness. In his theory, human development in terms of
perception is presented as the emergence of consciousness, which is not the
foundation in human existence. Consciousness is a "learned process based on
metaphorical language." Neurological studies in Jayne's presentation show
activity of the right temporal-parietal lobe during auditory hallucinations,
which corresponds to the language area of the left hemisphere. Studies of
patients whose cerebral hemispheres have been split show that they "function
as two independent persons." Bicamerals, or modern schizophrenics, perceive
hallucinations emanating from the right hemisphere as coming from outside
themselves. The right hemisphere of the brain is said to have a language
function, which people perceive as the "language of the gods." The excitation
of the right temporal lobe is therefore associated with an increased
religiosity or experience of God. The hallucinations here are often critical
in nature, such that the right hemisphere tends to "look down on the left
hemisphere." Through various assumptions, which will not be further explained
here, Jaynes therefore postulates a developing self-awareness through an
"external" influencing factor by the right hemisphere. Mintzberg similarly
names this influencing factor in his way of looking at the nature of
management by the right hemisphere, without explaining the development of
consciousness. The intuitive, strategic learning model postulated by Mintzberg
is given a framework by this way of looking at things, which is found in a
kind of "image of God" through the organizational structure. Therefore, the
objection in \cite{seminar} that "in this process [...] strange structures
[emerge] as described in "Mintzberg on Management," where "an orientation
structure is [given] in linguistic form that the management novices [have] to
follow and adopt, while these rules are not necessarily applied by the same
management gurus who teach the novices." is understandable. The distinction
already made above between practical management at Level 1, systematic
application of management experience at Level 2, and its academic study at
Level 3 is considered inconclusive by Jaynes' and Mintzberg's theses'
consideration of the left and right hemispheres of the brain because it cannot
have real-world application. Although Mintzberg highlights his previously
defined roles of the manager and their actions in the day-to-day management
process, no further reference is made to the external factors that allow the
manager's "right brain" to develop in the first place. The reason for public
relations in the form of rituals or ceremonies is not placed in any
sociocultural context that explains certain manners or beliefs. These are
taken as "given" and, according to Mintzberg's model, cannot change. If, in
the management process, a process external to the subject determines the
framework for action, how is this structured and what central force guides it?
According to Mintzberg, is a manager's success not dependent on his or her
ability, but on his or her external influence and ability to view social
patterns as well as "divine images"? For what reason can these patterns not be
passed on as conceptual designs to the next generation of managers? If the
intuitive process is considered a basic prerequisite for successful managing,
why not teach intuition?

\subsection{Ansoffs Criticism on Mintzbergs Strategy}

H. Igor Ansoff criticizes Mintzberg's definition of strategy by considering
the previously explained distinction between "planned" and "emergent" strategy
\cite{ansoff}. When a planned strategy should be applied is not explained, nor
under what conditions it should be carried out. Although Mintzberg describes
that "planned" strategies can sometimes be applied under stable environmental
conditions, when environmental conditions are stable and when "sometimes" is
is not further explained. What kind of environmental conditions Mintzberg
means (political, financial, logistical, personnel?) he does not elaborate.
Ansoff further shows that organizations generally have to deal with different
environmental conditions, constant or turbulent. In this context, turbulent
influences are a central driving force for strategy development, whereas
organizations under unchanging conditions succeed by incrementally developing
their strategies. Moreover, a company is even at a disadvantage when it
applies Mintzberg's emergent strategy under turbulent influences. Ansoff
further criticizes the confusion of cause and effect of Mintzberg's theses,
which, according to him, take the descriptive form of strategy development and
the manager's observations as the rationale for formulating prescriptions for
successful management. In making this assumption, Ansoff points to the fact
that there is no evidence that a strategy can be successful only after a
series of mistakes have been committed that first formed that
strategy. Planned diversification of strategies is, according to Ansoff,
financially more successful than trial and error of an "emergent"
strategy. Mintzberg, in his view, gives no basis for a failure to formulate an
explicit strategy when it is uncertain. Also, according to Ansoff, managers
are not either "certain" or "uncertain" about their strategies, but in reality
are always "between the extremes." Since Mintzberg already points out that
strategies are patterns from the past, Ansoff posits that strategies must be
present prior to events in order to positively influence them. In contrast to
"strategic learning", a rational learning model can show time-saving
alternatives and initiate the promising processes. Strategic mistakes can be
deliberately excluded and costs reduced. Mintzberg unfoundedly rejects the
rational learning model as a legitimate tool. Ansoff goes on to point out that
the decoupling of strategy formulation from implementation that Mintzberg
criticizes was already adapted to reality in the 1980s. Mintzberg focusses
only on sources around the 1950s or his own observations. Neither the
environmental conditions nor the throughput in which strategy development
takes place is described by Mintzberg. Furthermore, he does not give any
verification mechanisms to what extent his prescriptions about strategies can
be validated. The basic assumption about the universal applicability of the
strategic learning model with the interrelation of trial and error is not
questioned or substantiated by Mintzberg at any time. According to Ansoff,
emergent strategy development is a valid prescription for management only
under "heightened environmental conditions", which he does not explain in
further detail.

\subsection{Management as a Technical System}

If one regards the process of the management from the view of a technical
system, Shpakovsky points out in \cite{shpakovsky} that this is defined by a
use, which is given from the outside and is fulfilled by the system. The
technical system is also distinguished in the description form, which
represents the interpersonally communicated expectations, and the
implementation form, which produces experienced results \cite{graebe:2020}.
Mintzberg does not emphasize this difference in his management theory. In this
context, he neither specifies an exact benefit to the system, nor how this
should be fulfilled by the system. Based on the VDI standard 4521, which
emphasizes the technical system as a man-made totality of several interacting
elements that fulfill a purpose, Mintzberg, similar to the VDI, gives only the
effect of the system of managers as a model, which correspond to his
observations, but does not give an explanation of the implementation form in
real-world form, which is made possible by the management process. Individual
elements in technical systems and the totality of components are considered by
Mintzberg in combination with organizations.

\section{Organisations}

\subsection{Organisations as Configurations}

Mintzberg highlights the definition of an organization as a combination of
different configurations. The concepts of components and participants in
organizations, the different organizational coordination mechanisms, the basic
types of organizations, the fundamental forces in organizations, and the life
cycle of organizations are mentioned in the context of these configurations.

The components and participants of an organization are divided into six
different units. The strategic top is formed by full-time managers. These are
responsible for the management and control of the entire organization. Below
them is the middle line management with the managers for the operators and the
managers for the strategic top. Supporting units such as the cafeteria,
mailroom, legal department and public relations regulate external
communications and ensure internal process flows. The Operators, the
operational core, do the main work in terms of production and services and
represent the main part of the organization. The technostructure is staffed by
analysts who perform various administrative tasks such as formal planning and
control processes. They also pass on certain standardizations to the
operators. Lastly, there is an ideology in the form of tradition or beliefs to
differentiate from other companies or to project charisma to the outside
world. The components are illustrated in Figure 1.

\begin{figure}[ht]
\centering
\includegraphics[scale=0.6]{basic_types.png}
\caption{Basic types of organizations: Mintzberg on Management, Henry
  Mintzberg: 1991 p.110}
\label{fig:basic_types}
\end{figure}

\subsection{Coordination Mechanisms}

Coordination mechanisms amount to six different types (illustrated in Figure 2
in Appendix). Mutual agreement is an informal agreement between employees -
the operators. Direct control is carried out by the strategic top and passed
on to the operators. Standardization of the work process is passed on to the
operators by the technostructure in the form of agreements in order to achieve
consistency in the work processes. In contrast, the standardization of the
results is the consistency in the presentation of the work results. Then
standardization of skills establishes coordination in the form of training of
operators. Workers are thus aligned with each other. Last, alignment of
beliefs of the organization occurs through standardization of norms.

\subsection{Basic Types of Organisations}

The basic types of organizations are divided by Mintzberg into seven different
units (listed in Table 1). The entrepreneurial organization is characterized
by a simple, informal and flexible structure, usually consisting of a simple
strategic top and the operators. The other components are not yet developed in
this configuration. The organizations usually start in this structure and are
surrounded by a simple but dynamic environment with a lot of
competition. Certain leadership qualities such as authority and charisma are
prerequisites for leaders in this configuration. Even under crisis and change,
this configuration can occur, needing a visionary view and a rough roadmap
that keeps the organization alive. The company has a specific mission, which
also makes it vulnerable and carries risk. Most companies fail in this
configuration because the mission does not always match reality. The machine
organization is characterized by permanence in its configuration. Bureaucracy
is its characteristic. The technostructure is crucial because it is where the
work processes for the entire organization are established. The mode of
operation is clocked and planned through. The environment is stable and the
configuration is usually found in larger, established companies. Strict rules
give the configuration its stability even in changing conditions. Efficiency
and reliability characterize the image of the organization, but the fixation
on control can cause the organization to fail in upheavals. The diversified
organization is characterized by different divisions and tends more towards
the aforementioned machine organization. An autonomous management envelops the
different divisions with their strategy. This form is mostly found in
different market segments and has already reached a kind of maturity. It is
increasingly found in governments or public sectors. As a result, they tend to
focus on economic rather than social goals. Divisions can retain their
individuality by pursuing their own strategies as long as they fit the overall
strategy. The organization can distribute risk through its form and include or
omit different strands of the business. The disadvantages are that innovations
can only be implemented very slowly and they can also act in a socially
irresponsible manner. The organization of the professionals is bureaucratic
but decentralized in the form of various institutes, with control resting with
the professionals. The context of the organization is complex but
stable. Strategies are mostly heterogeneous. Decisions are made based on the
judgment of the professionals or collective decisions. The advantages of the
democratic decision-making process often also lead to coordination
problems. Innovations cannot be implemented quickly because the professionals
rely on their knowledge and cannot reorient themselves. In contrast, the
innovative organization is very dynamic and organic and is built by experts in
multidisciplinary teams. Coordination is usually through direct arrangements
and the strategic top and the operators are not clearly separated. These
organizations are mostly young and focus on the learning and growth
process. The main goal is innovation, which comes at the expense of
efficiency. The goal is more important than economic profitability. An
external operational core may be brought in. The missionary organization is
defined by its ideology. Differentiation from other organizations is its
unique selling point and is usually accompanied by charismatic leadership. The
configuration may overlap with others, especially in entrepreneurial,
innovative, or even machine organizations or organizations of
professionals. The political configuration is characterized by conflict and
can overlap other configurations, but is capable of holding its
own. Coordination is usually present in the form of a power play, which leads
to confrontation and uncertainty. Change is necessary and comes to light
through political action.

\begin{table}[ht]
\begin{tabular}{| p{3cm} | p{3.8cm} | p{3.8cm} | p{3.8cm} |}
\hline Configuration & Primary coordination mechanism & Key part of the
organization & Type of dezentralization \\ \hline \hline Entrepreneurial &
Direct control & Strategic top & Vertical and horizontal
centralization\\ \hline The machine organization & Standardization of work
processes & Technostructure & Limited horizontal decentralization \\ \hline
Professionals & Standardization of skills & Operational core & Horizontal
decentralization\\ \hline Diversified & Standardization of outputs & Middle
line management & Limited vertical decentralization\\ \hline Innovative
organization & Mutual coordination & Supporting units & Selective
decentralization\\ \hline Missionary organization & Standardization of
standards & Ideology & Decentralization\\ \hline Political organization & None
& None & Various\\ \hline
\end{tabular}
\caption{Basic types: Mintzberg on Management, Henry Mintzberg: 1991 p.120}
\label{Tab:types}
\end{table}


\subsection{Fundamental Forces in Organisations}

According to Mintzberg, the configurations should be seen as an integrated
framework of fundamental forces (see Figure 3 in Appendix). Each force must be
connected to a counterforce to keep the organization alive. Entrepreneurial
organizations tend to have a direct force initiated by the leader. The machine
organization is usually characterized by efficiency. Professionals want to
expand their skill set and break free from control. Diversified organizations
tend to concentrate power. Innovative organizations want to bring about change
and adaptation through the learning process. An organization's
self-destruction is avoidable only by applying the principles of cooperation
and competition, with policies and ideological beliefs in opposition.

\subsection{Life Cycle Model of Organisations}

Mintzberg contextualizes the above configurations with their associated forces
in terms of a life cycle model (see Figure 4 in Appendix). Configurations are
divided into formation stage, development stage, maturity stage, and decline
stage. Transitions between configurations are initiated by political
confrontations or tend to revitalization, a renewal of the configuration. As
mentioned above, organization mostly start in the form of entrepreneurial
organization in the formation stage. These organizations are characterized by
a mission and are maintained as long as the leader is in charge. The demise of
the organization due to the above reasons is symbolized by a tombstone in each
case. Self-correction is not present in this configuration. The missionary
organization is most likely as the following configuration, because there the
charisma of the leader is installed as beliefs. Further following there are
two possibilities for an advancement of the entrepreneurial configuration, if
it is based on expert knowledge: Either it tends to the innovative
configuration, when its mission shapes its progress through creativity, or to
the professional organization, where a standardization of skills is
preferred. On the other hand, the entrepreneurial configuration can become a
machine configuration, divided into instrumental and closed machine. The
instrumental machine is created by the takeover of an external influencer or a
power takeover from outside the organization. When internal management is
strong enough to assert itself, the entrepreneurial organization transforms
directly into a closed machine organization. The transition from missionary
configuration to closed machine occurs through the ideologized inefficiency of
the organization. From instrumental to closed machine organization, the
transition is characterized by an adjustment of internal stability. Starting
from the closed machine organization, it then tends to diversify as it expands
in size and influence. Bureaucratic status is placed at a new level and
revitalization resembles a proliferation of individual divisions. With little
external control, the closed machine organizations and the professional
organization tend to become politicized, which can lead to moral decay. In the
innovative configuration, political processes are seen as a temporary
difficulty which can lead to a turnaround or revitalization. Lastly, the
political configuration usually leads to the demise of these if it lasts too
long, as organizations cannot sustain long periods of conflict.

\subsection{Control Mechanisms of Organisations}

In the context of the life cycle model, Mintzberg already highlights the
hypothesis that organizations are kept alive for too long, which could be
replaced by other organizations and would allow a more sensible, productive
use of free resources. He critiques this form of control through mechanisms
(see Figure 5 in Appendix) realized by government and economic policies. "We
create organizations so that they serve us. But somehow they also force us to
serve them." \cite[p. 307]{Mintzberg}. Mintzberg distinguishes between eight
different control mechanisms: nationalization here means government
interference when a task is recognized as important but not covered by the
private sector. In addition, nationalization is considered when an
organization's tasks are so closely related to state activities that it can be
run as a "direct arm of the state." Democratizing is understood as
facilitating the expansion of corporate governance, with power being
constitutionally decentralized. Regulating organizations occurs through
obligations from regulators and courts to the organization to perform certain
activities. Limits can also be imposed on them, leaving internal control with
managers. Exerting pressure is meant as encouraging organizations to adjust
their behavioral fundamentals. For example, activist campaigns aim to make
organizations act in a socially responsible way. Trust means that business
leaders are trusted to observe social goals on their own - simply because it
is a "good thing" to do. The difference with ignoring is that a strong economy
is seen as a prerequisite for achieving social goals. Economic goals cover
social goals as well. Incentives can be donated, mostly in the form of
subsidies for organizations. These receive rewards through
constraints. Restoring is equating freedom through entrepreneurship, with
economic profit as an indicator of good behavior. Mintzberg notes that
financial incentives do not belong where a company has caused a problem, but
has the ability to solve a problem caused by others. One should not reward a
company for acting "badly."

\subsection{Comparison of System Development}

Alexander Solodkin in his publication "Discovery of key problem causes in
Organizations" already presents distortions and noise as the main problems in
organizations. According to Solodkin, problems are the gap between the current
and the desired system state \cite{solodkin}. Therefore, both states should be
described as accurately as possible. This difference has already been
clarified in \cite{graebe:2021} in the evolution of systems over time. Here,
an additional distinction is made between the states and the
transformations. Starting from the present system, the system as demanded is
achieved by a desired transformation. The true transformation, which takes
place in the implementation, leads to the system, which actually arises. A
contradiction arises between the ideal line of development and the real line
of development. In the TRIZ concept of the Ideal Final Result (IFR), according
to Solodkin, the desired system state can be described. In Mintzberg's
expositions of the configurations of organization, the aspired system states
and the actual system states are not distinguished. The transitions between
configurations are presented by Mintzberg as laws that are passed through
purely political processes. Since we are dealing with socioeconomic and
cultural systems, respectively, according to Solodkin, the processes are
subject to various biases and noises that Mintzberg ignores in his description
of configurations and the life cycle model of organizations. While Mintzberg
cites various forces that influence the organization, these all arise from
within the organization itself. External influences such as the capital
market, ecological influences, or material do not find sufficient description
in Mintzberg's model. The implementation form of an organization, which brings
it to the next configuration to expand or overcome obstacles, is not
specified. The stages of the organization seem to exist as givens by Mintzberg
and are not given a developmental dimension outside of political
confrontations. The difference between development of the system, development
of the components of the system (here: actors, managers, technostructure,
etc.) and relationships within the system is not put into context. Throughput,
deliberate processes of change, or activities by management are not included
in the life cycle model. Mintzberg's two main assumptions, first, that
organizations exist as stable and persistent forms, "but their status changes
regularly" \cite[p. 287]{Mintzberg} and second, that forms are always in
different "stages of life," i.e., the organization is in the stages of birth,
growth, maturity, and decline or death, describes the changes solely in the
power constellation of organizations. In contrast, structural change, which is
consciously or unconsciously shaped by management, is only present in
descriptive form and, according to Mintzberg, seems to be more or less subject
to chance. The problem analysis, which would make the transformations
predictable, contradicts Mintzberg's emergent strategy development and does
not find application in the life cycle model.

\section{Conclusion}

For a sufficient description of management structures and the associated
application scenarios in different organizational structures, Mintzberg's
theses are insufficient in an academic setting. A system theoretical approach,
which brings forth the context of an organization in description and
implementation level, can only be seen from the perspective of a subjective
transformation to "better management" through Mintzberg's presentation of
emergent strategy development. The analogies presented are not so much
presented as a scientific analysis of processes of management, but are
intended to highlight an individual view of the social organization
itself. Mintzberg aims to take a socially critical stance through an
experiential description of processes in everyday work that relies less on
actual academic inquiry and instead views the concentration of power through
organizational structures in society through the "eyes of the manager." In his
models, he does not so much expose a framework for action that looks at system
processes at their core, but ultimately criticizes the politicized basis for
successful management that he believes has "made society uncontrollable"
\cite[p. 331]{Mintzberg}. His accounts are therefore not to be seen as
sociotechnical investigations and are therefore unsuitable for the
institutionalized use and associated scholarly reflection mentioned at the
outset. On a tactical, subjective level, Mintzberg's view may nevertheless be
helpful, or as he describes it in the preface to his book, "This book is
written for those who spend their public lives with organizations and recover
from them in their private lives."

\newpage

\begin{thebibliography}{Gra21b}

\bibitem[Ans91]{ansoff} \textsc{Ansoff}, H.~I.: \newblock Critique of Henry
  Mintzberg‘s \emph{The Design School: Reconsidering the Basic Premises of
  Strategic Management}.  \newblock {In: }\emph{Strategic Management Journal}
  12 (1991), pp. 449--461.

\bibitem[Bur54a]{burns} \textsc{Burns}, Stewart: \newblock \emph{The
  Directions of Activity and Communication in a Departmental Executive Group}.
  \newblock Human Relations, 1954.

\bibitem[Bur54b]{stewart2} \textsc{Burns}, Stewart: \newblock \emph{Managers
  and Their Jobs}.  \newblock Human Relations, 1954.

\bibitem[Dav57]{davis} \textsc{Davis}, Robert~T.: \newblock \emph{Performance
  and Development of Field Sales Managers}.  \newblock Cambridge: Division of
  Research, Harvard Business School, 1957.

\bibitem[Gra20]{graebe:2020} \textsc{Graebe}, Hans-Gert: \newblock Technical
  Systems and Their Purposes.  \newblock In: TRIZ-Anwendertag 2020 (Oliver
  Mayer, ed.), Springer Verlag 2021, S. 1-13. \newblock
  \url{http://dx.doi.org/10.1007/978-3-662-63073-01}.

\bibitem[Gra21a]{graebe:2021} \textsc{Graebe}, Hans-Gert: \newblock On the
  concept of system in the Theory of Dynamical Systems.  \newblock (2021).
  \newblock
  \url{https://github.com/wumm-project/Seminar-S21/blob/master/Lecture/TDS.md}.
  
\bibitem[Gra21b]{seminar} \textsc{Graebe}, Hans-Gert: \newblock Seminar Notes:
  Summer Term 2021.  \newblock (2021).  \newblock
  \url{http://www.dorfwiki.org/wiki.cgi?HansGertGraebe/LeipzigerGespraeche/2021-07-23}

\bibitem[Jay]{Jaynes} \textsc{Jaynes}, Julian: \newblock Summary of Evidence
  for the Bicameral Mind Theory.  \newblock
  \url{https://www.julianjaynes.org/about/about-jaynes-theory/summary-of-evidence/'}

\bibitem[Min73]{Mintzberg2} \textsc{Mintzberg}, Henry: \newblock \emph{The
  Nature of Managerial Work}.  \newblock New York: Harper \& Row, 1973.

\bibitem[Min91]{Mintzberg} \textsc{Mintzberg}, Henry: \newblock
  \emph{Mintzberg über Management}.  \newblock Gabler Verlag, 1991.

\bibitem[Que56]{guest} \textsc{Quest}, Robert~H.: \newblock \emph{Of Time and
  the Foreman}.  \newblock Personnel, 1956.

\bibitem[Shp]{shpakovsky} \textsc{Shpakovsky}, Nikolay: \newblock Human and 
  technical systems.  Online, 2003. \newblock
  \url{https://wumm-project.github.io/Texts/Shpakovsky/mts-ru.pdf}

\bibitem[Sol]{solodkin} \textsc{Solodkin}, Alexander: \newblock Discovery of
  Key Problem Causes in Organisations.  Online, 2020.\newblock
  \url{https://wumm-project.github.io/Texts/Solodkin-TDS2020-en.pdf}

\bibitem[Ste51]{stewart} \textsc{Stewart}, Rosemary: \newblock \emph{Managers
  and their Jobs}.  \newblock New York: McMillan, 1951

\end{thebibliography}

\clearpage

\section{Appendix}

\begin{center}
\includegraphics[scale=0.7]{koordinationsmechanismen.png}\\
{Coordination mechanisms: Mintzberg on Management, Henry Mintzberg:
  1991 p.113}
\label{fig:coordination}

\includegraphics[scale=0.7]{forces.png}\\
{Coordination mechanisms: Mintzberg on Management, Henry Mintzberg:
  1991 p.264}
\label{fig:forces}

\includegraphics[scale=0.7]{lifecycle.png}\\
{Life cycle model: Mintzberg on Management, Henry Mintzberg: 1991
  p.288}
\label{fig:lifecycle}

\includegraphics[scale=0.7]{control.png}\\
{The conceptual horseshoe: Mintzberg on Management, Henry Mintzberg:
  1991 p.312}
\label{fig:control}
\end{center}
\end{document}
