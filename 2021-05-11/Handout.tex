\documentclass{scrartcl}

% Formatting
\usepackage[utf8]{inputenc}
\usepackage[margin=1in]{geometry}
% \usepackage[titletoc,title]{appendix}
\usepackage{amsmath,amsfonts,amssymb,mathtools}
\usepackage{graphicx,float}
\usepackage[english]{babel}
% Title content
\title{System Thinking and Management}
\subtitle{Russel L. Ackoff}
\author{Jannis de Riz}
\date{\today}

\begin{document}
\maketitle
% Abstract
\section{Introduction}
The \glqq System movement\grqq as Schedrovitsky calls it \cite{shchedrovitsky1981principles} originated for the most part between the 1950s and 1960s with the notion of general systems theory by Ludwig von Bertalanffy and system dynamics (industrial dynamics) by Jay Forrester  \cite{von1956general} \cite{forrester1994system} \cite{forrester1997industrial} \cite{forrester1968industrial}. 
Russel Lincoln Ackoff was an american Scientist, Consultant and one of the contributors to the aforementioned movement. He received his Bachelors degree in architecture in 1941. He later went on to teach mathematics and philosophy at the Wayen State University and became a Professor for Operations Research at the Case Institute of Technology. His biggest contribution to the sciences was the socio-systemic approach to organization theory. His works concentrated on systems thinking and its implications for operations research and management \cite{ackoff1972note} \cite{ackoff1989data} \cite{sengupta1965systems} \cite{ackoff2006few}. \\ 
He pointed out that after the second world war, western culture shifted into a systems age where everything had to be taken apart and analyzed. Thinking about purpose was considered unproductive and meaningless \cite{ackoff1972note}. He hypothesized that all failure in management resulted from that purely analytical line of thought and not thinking systemically \cite{shortspeech}. 

\section{Systems as understood by Ackoff}

Ackoff puts forward the notion that, in contrast to the common view, a system is the product of its parts interactions rather then their sum. He further distinguishes between three types of systems, there being the mechanical, the organismic and the social system \cite{ackoff1994systems}. 
He defines a System as follows: \\ 

\begin{quote}\itshape{\glqq A System is a whole consisting of two or more parts(1) each of which can affect the performance or properties of the whole,(2) none of which can have an independent effect on the whole, and (3) no subgroup of which can have an independent effect on the whole. In brief, a system is a whole that cannot be divided into independent parts or subgroups of parts. \grqq 
	}
\end{quote}	

There are several implications that arise from this definition. For example, a system can consist of several subsystems that in themselves form a system. Also, all parts of a System are interdependent which follows from (2). That means changing one part can never be seen in isolation. A change in one, say subsystem, is always accompanied by at least on counteraction. In other words, all parts and their contribution to the system has to be seen in the context of at least one other part.
Parts without which a system can not perform its function are called essential.
As already mentioned Ackoff discriminated three types of systems.

\paragraph{Mechanical} systems are open or closed, as in, they can or cannot interact with their environment. Mechanical systems have no purpose on their own, instead they serve the purpose they where designed for. A car has no purpose on its own but serves the purpose of transportation that it was designed for.  


\paragraph{Organismic Systems}
The organismic system has one goal or purpose that is inherent to it. As humans our body-systems purpose is assuring to survive, or to continue being. Each individual part of our body in contrast has no purpose but a function. Organismic systems are open and therefore react or interact with the environment.



\paragraph{Social Systems}
Ackoff defines social systems as follows. They {\itshape{ \glqq are open systems that (1) have purpose of their own, (2) at least some of whose essential parts have purpose of their own, and (3) are parts of larger (containing) systems that have purpose of their own. \grqq}} \cite{ackoff1994systems} 
\\

Those three representations of a system are both concept and entity. This enables to think of any system in terms of any of these types \cite{ackoff1994systems}. 


\section{Management}
Most definitions of management are based on an corporation/business view of Management such as the cambridge dictionary when it sates: management is\\
\begin{quote}\itshape{ \glqq the control and organization of something, esp. a business and its employees \grqq
} \cite{ackoff1994systems}
\end{quote}

The German wikipedia page has a more general definition of the term management which states: Management (lat. manus $\rightarrow$ hand, lat. agere $\rightarrow$ guide/direct) :
\begin{quote}\itshape{ \glqq is every goal oriented human motion thats pursued after economical principles and guides, organizes and plans in all aspects of live. \grqq
} \cite{ackoff1994systems} 
\end{quote}

Ackoff puts forward a chronologically evolving view of enterprises, and thereby direct implications towards their management, from industrial revolution to present(1972) and from mechanistic over organismic to social.

\subsection {Enterprise as a Machine}
If seen as a machine, the workers are the machines parts. They are interchangeable as most functions did not require some specifically complex skill during the beginning of the industrial revolution. The owner (manager) of the machine was \textit{ \glqq all-powerfull\grqq}  and could do whatever he or she (in those times mostly he) pleases. If a part broke, it could easily be replaced. And parts did break as there was virtually an unlimited supply of spare parts and therefore there use was reckless.

\subsection{Enterprise as an Organism}
When the 19s century came to an end the mechanistic view of an enterprise was replaced by a more organismic concept. Due to public companies whos ownership could be bought in parts at the stock market the notion of the owner became an abstraction. The jargon used to describe an enterprise as corporation (lat. body) reflected this abstraction and therefore increase in complexity. Increasing complexity of ownership was accompanied by increasingly difficult work tasks that required more skilled workers. These were not to be found as easily as in the \textit{\glqq enterprise as machine\grqq} age. That positioned them, to varying degree, closer to the essential parts rather then replaceable ones.
Managers had to treat them as humans with their own purposes and wishes in order to get good results from the workforce.

\subsection{Enterprise as a Social System}
The further increasing complexity of the system they managed(corporation) as well as the system they managed their system in(society, state), managers had to adapt a social view on their task. They now had to be concerned with all three aspects of a social system as described in the corresponding paragraph. They (1) had to define a purpose for the company they where leading, (2) take care of the purposes and wishes of the people (systems) that where contributing to those first purposes and (3) lastly place that entire construct within the social system they where part of.

\section{Analytical versus Synthetical Management}
Ackoff states in 1972 \cite{ackoff1994systems} that
\begin{quote}
	\itshape
	managers are educated to believe that a social systems's performance can be improved by improving the performance of each of its parts taken separately - that is, if each part is managed well, the whole will be. This is seldom if ever the case, because parts that appear to be well managed when viewed separately seldom fit together well.
\end{quote} 

\paragraph{Analysis} is taking the whole apart and concentrating on managing every part individually. Understanding a system can not be done by analysis of its parts. The function of each individual part can tho.
\paragraph{Synthesis} instead is putting the system together with other systems (parts) and properties of that supersystem are derived in order to understand the function of the initial system of interest.


There is no explanation within the cars that explains why british cars are left driven in contrast to american or other european ones. The explanation lies in the system these cars are part of. 
\section{Problems and Messes}
Ackoff states there are essential treatments (managements) to problems or messes. 
\paragraph{Absolution} ignores the Problem.
\paragraph{Resolution} can be seen as a quick fix. It is an approach that results in a situation that merely satisfice. Its focus is on the very specific problem rather then the general mechanism behind it.
\paragraph{Solution} is within the given context the optimum. It is led by a research approach and focuses on the general aspects of the problem.
\paragraph{Dissolution} redesigns the entity or the environment where the problem arose. This enables for a future state that is superior to the best possible in the current one. It focuses on generality and uniqueness equally and uses whatever technique seem to be fit.

\section{Knowledge versus Understanding}
\paragraph{Knowledge} comes from analysis. It is knowing HOW something works.
\paragraph{Understanding} comes from synthesis. It is to understand WHY something behaves or works as it does.

\newpage

\bibliographystyle{acm}
\bibliography{bib}
\end{document}
