\documentclass[a4paper,11pt]{article}
\usepackage{a4wide,graphicx,url,enumitem}

% Formatting
\usepackage[utf8]{inputenc}
\usepackage[english]{babel}

\parindent0pt
\parsep 3pt

% Title content
\title{Selected Aspects of Russell L. Ackoffs System\\ Thinking and Management
  in the Context of the\\ WUMM Project's System Theory\\[2em] \large Seminar
  Assignment for the Module 10-202-2312\\ Applied Computer Science
  \emph{Co-operative Planning and Action}}

\author{Jannis de Riz}

\date{September 30, 2021}

\begin{document}
\maketitle
\vfill
\tableofcontents
\newpage
% Abstract
\section{Introduction}
This assignment is presented as an extension of a previous talk held in the
seminar "Co-operative planning and action" as part of the Applied Computer
Science module 10-202-2312. The seminar is held as part of the WUMM project
\cite{wumm_project}. There is not one final definition of a system but rather
a vast collection of many different sources from different times and with a
different emphasis. Therefore, one can not always assume that every
participant in a discussion is argumenting based on the same concept of a
system. In each individual frame of reference certain notions might carry a
slightly different character. It is even possible that very similar concepts
can be expressed using different terms, as will be shown later. It is for that
reason, that a shared understanding of concepts and notions needs to be
established first in order to communicate ideas consistently.

The "System movement" as Schedrovitsky calls it
\cite{shchedrovitsky1981principles} originated for the most part between the
1950s and 1960s with the notion of general systems theory by Ludwig von
Bertalanffy and system dynamics (industrial dynamics) by Jay Forrester
\cite{von1956general} \cite{forrester1994system}
\cite{forrester1997industrial} \cite{forrester1968industrial}.  Russel Lincoln
Ackoff was an american Scientist, Consultant and one of the contributors to
the aforementioned movement. He received his Bachelors degree in architecture
in 1941. He later went on to teach mathematics and philosophy at the Wayen
State University and became a Professor for Operations Research at the Case
Institute of Technology. His biggest contribution to the sciences was the
socio-systemic approach to organization theory. His works concentrated on
systems thinking and its implications for operations research and management
\cite{ackoff1972note} \cite{ackoff1989data} \cite{sengupta1965systems}
\cite{ackoff2006few}.

He pointed out that after the second world war, western culture shifted into a
systems age where everything had to be taken apart and analyzed. Thinking
about purpose was considered unproductive and meaningless
\cite{ackoff1972note}. He hypothesized that all failure in management resulted
from that purely analytical line of thought and not thinking systemically
\cite{shortspeech}.

This assignment's aim is to clarify the relation between Russell Ackoffs
approach to systems theory to the one developed in the context of the WUMM
project. Therefore, both viewpoints on the notion of a system are presented
and compared. Further, a selection of two system theoretic sub topics
concerning co-operative planning and action is presented and implications from
both frames of reference (Ackoff, WUMM) are discussed. The first detailed
analysis will be on the notions of Analysis and Synthesis in contrast to
immersion and submersion. In Ackoffs words: \textit{understanding HOW
  something works} (Analysis) and \textit{understanding WHY something works}
(Synthesis) \cite{ackoff1994systems}. The second matter will be the topic of
Problem solving and transition pathways and how the understanding of a system
influences these.


\section{Comparison of Russell Ackoffs and the WUMM Projects System
  Defintions} 
\label{system_theory}
In this section, Ackoffs understanding of systems is examined from the
perspective of the WUMM systems theory approach. Therefore, both System
Theories are presented. Then similarities as well as differences are pointed
out and their overall interconnectedness is discussed. It should be noted that
Ackoff stated his notion of a system in the 1970s whereas the WUMM projects
definition of a system is from 2019.

\subsection{Systems as understood by Ackoff}

Ackoff puts forward the notion that, in contrast to the common view, a system
is the result of its parts \textit{interactions} rather then their sum. This
is actually a lot closer to Aristoteles initial statement \emph{…the totality
  is not, as it were, a mere heap, but the whole is something beside the
  parts…  } which led to the common notion of \emph{the whole is more than the
  sum of its parts } which might not emphasize the emergent phenomenon as much
as the initial statement of Aristoteles. He further distinguishes between
three types of systems, there being the mechanical, the organismic and the
social system \cite{ackoff1994systems}.  He defines a system as follows:
\begin{quote}
  A system is a whole consisting of two or more parts
  \begin{itemize}
  \item[(1)] each of which can affect the performance or properties of the
    whole,
  \item[(2)] none of which can have an independent effect on the whole, and
  \item[(3)] no subgroup of which can have an independent effect on the
    whole. In brief, a system is a whole that cannot be divided into
    independent parts or subgroups of parts.
  \end{itemize}
\end{quote}	

There are several implications that arise from this definition. For example, a
system can consist of several subsystems that in themselves form a system.
Also, all parts of a system are interdependent which follows from (2). That
means changing one part can never be seen in isolation. A change in one, say
subsystem, is always accompanied by at least one counteraction. In other
words, all parts and their contribution to the system has to be seen in the
context of at least one other part.  Parts without which a system can not
perform its function are called \emph{essential}.  As already mentioned Ackoff
discriminated three types of systems.

\paragraph{Mechanical systems.}
Mechanical systems are open or closed, as in, they can or cannot interact with
or be influenced by their environment. Newton saw the universe as a closed
system. In contrast the earth was seen as an open system as its trajectory is
influenced by its surroudning stars and planets. Mechanical systems have no
purpose on their own, instead they serve the purpose they where designed for.
A car has no purpose on its own but serves the purpose of transportation that
it was designed for.

\paragraph{Organismic systems.}
The organismic system has one goal or purpose that is inherent to it. As
humans our body-systems purpose is assuring to survive, or to continue being.
Each individual part of our body in contrast has no purpose but a function.
Organismic systems are open and therefore react to, or interact with, the
environment.

\paragraph{Social systems.}
Ackoff \cite{ackoff1994systems} defines social systems as follows. They "are
open systems that
\begin{itemize}[noitemsep]
\item[(1)] have purpose of their own, 
\item[(2)] at least some of whose essential parts have purpose of their own,
  and 
\item[(3)] are parts of larger (containing) systems that have purpose of their
  own."
\end{itemize}

Those three representations of a system are both concept and entity. This
enables to think of any system in terms of any of these types
\cite{ackoff1994systems}.  It should be noted that Ackoff develops these three
perspectives in front of a historical context, where one developed after
another in consecutive order as stated. This implies that both organismic as
well as social systems are to some extend descendents of the mechanical system
view.

\subsection{Systems as understood by the WUMM project}%
\label{sec:systems_as_understood_by_the_wumm_project}

The WUMM project orients its system view on the TRIZ methodology
\cite{altshuller200240}. The (Business) TRIZ methodology is focused on
technical and socio-economical systems. Although there is no unique system
definition in TRIZ the WUMM project defines a system in its ontology
\cite{triz_system} as follows:
\begin{quote}
  A system is a set of elements that are in relationship and connection with
  each other and that constitute a well defined unity, an integrity. The
  necessity of the use of the term ”system” occurs when it is required to
  emphasize that something is large, complex, immediately not wholly
  comprehensible, but at the same time a unified whole. Unlike the notions
  ”set” or ”aggregate”, the concept of a system emphasizes the ordering, the
  integrity, the regularity of construction, functioning and development.
\end{quote}	

A definition by itself is prone to misunderstanding due to the reasons
mentioned in the introductory section. Therefore, its embedding in further
explanations from \cite{grabe2020seminar} will be discussed briefly.  The
authors in \cite{grabe2020seminar} are developing their notion of a system in
a dynamic socio-economical and socio-technical context. Much like Ackoff, the
TRIZ terminology distinguishes different perspectives on a system. In
\cite[p.~69]{grabe2020seminar} the authors mention technical-, engineering-
and man-made systems. Buth those are not the only system notions discussed in
the WUMM project.  This general approach of establishing a system definition
that serves as an architectural skeleton which then is concretized by further
explanations is the same as in Ackoffs writings. Yet, the WUMM projects
elaborations are more human focused then Ackoffs perspective in general.
Ackoff develops the social system, over a mechanical and then in general
organismic system. Both do not have a clearly defined human aspect. Of course
the mechanical system can be build by humans, but that is not a central part
of the concept. The TRIZ system elaborations mentioned in
\cite[p.~69]{grabe2020seminar} are in contrast all including the human as a
central entity in one or the other way.

\subsection{Comparison of definitions}
\label{subsec:Definition comparison}
Both notions of a system share many commonalities.  One could summarize both
definitions in three major points.  
\begin{itemize}
\item[(1)] First and foremost both state that systems have a set of
  entities (parts, elements). 
\item[(2)] These entities interdepend (have relations, are connected) on each
  other.  
\item[(3)] The system has integrity, forms a unity, in other words
  \emph{cannot be divided}.
\end{itemize}
Now, as the common characteristics are established, what are the differences
between both understandings?  Statement (2) of Ackoffs definition is
unmirrored in the WUMM system definition. No single part can have an
independet effect on the whole and neither can any subgroup of parts. From a
graph theoretical view this is the case for connected graphs, as there is a
path (way of interaction) from any one point to any other. This strict
constraint articulates Ackoffs emphasis. In contrast the distinguishing
emphasis in the WUMM system definition lies on the overall, at first
uncomprehensible, complexity of a system. Although it has order, integrity and
regularity of construction, functioning and development. In other words, WUMM
emphasises "uncomprehensible" complexity (emergence) despite comprehensible
description. And Ackoff emphasises the cause of that complexity, which is
strict interdependency. We know from the three-body problem that systems with
only 3 elements or parts already show complex and chaotic behavior and cannot
be analytically determined, in other words no closed form can be found but
only numerical solutions. To summarize, both definitions are fairly well
aligned, but in WUMM putting emphasis on the observed behavior and in Ackoffs
case on the cause of such observed behavior.

\section{System-Theoretical Implications of System Perspectives for
  Co-operative Planning and Action} 
\label{system_view_and_management}
Following the seminars title \emph{Co-operative Planning and Action} this
section discusses how the systemtheoretical framework can influence the
process of designing (planning) and managing (acting on) an actual
\textit{real world} system. Specifically the concepts of analysis and
synthesis as stated by Ackoff are compared to the submersive and immersive
approach to systems in WUMM. Consecutively, possible implications of these
concepts for planning and action are discussed.

\subsection{Analysis and synthesis or submersion and immersion}
\label{sec:analysis_and_synthesis_or_submersion_and_imersion}
In this section Ackoffs notions synthesis and analysis are presented and set
into relation with the WUMM concept of immersive and submersive system theory
\cite[p.~21]{grabe2020seminar}.  The terms submersion and immersion have
distinct meaning in many different fields \cite{van2005new}
\cite{hakuta1987synthesis}. Therefore, the notions have to be explained in
this system theoretical context.

\paragraph{Synthesis}
is putting the system together with other systems (parts) and properties of
the supersystem are derived in order to understand the function as inputs and
outputs of the initial system of interest.

\paragraph{Analysis}
instead is taking the whole apart and concentrating on managing every part
individually. Understanding a system cannot be reduced to the analysis of its
parts, but the function of each individual part is an important ingredient for
the overall understanding of a system.

\paragraph{Immersive system theory}
is characterized by embedding and is in this context understood as the direct
sum of system components.

\paragraph{Submersive system theory}
is characterized by cascaded complexity reduction and projection of system
components. It is in this context to be understood as the product of system
components, and thereby does not equal to the sum of its parts.  
\bigskip

To now link those two terminologies and concepts, one can think of immersion
and synthesis to be a pair of concepts. Understanding is comprehension of
\textit{why} something works and is based on synthesis in Ackoffs writings. It
focuses on the sum (direct sum from WUMM immersion) of the inputs and outputs
of system components which define its role in the supersystem.

Analysis and submersion is the other pair. Analysis yields knowledge in
Ackoffs studies and is comprehension of \textit{how} something works. It
focuses on emergent phenomena and is the product (direct product from WUMM
submersion) of the system components interactions.

Ackoff writes, and one can easily follow this line of argument, in order to
manage properly these views requires to be followed but a combination of at
least two as they both shine light on \emph{different} aspects of the system.
It should be noted that this approach of analysis and synthesis or submersion
and immersion can be exercized in a recursive fashion. So when analysis is
conducted of any given system $S$ consisting of $S_i$ and $S_j$ their
respective parts can be evaluated from an immersive context such that we ask
what are the I/O properties (direct sum) of e.g $S_{ij}$ and $S_{ik}$ that
contribute to the purpose of $S_i$.

The authors in \cite[p.~21]{grabe2020seminar} write that the theory of dynamic
systems is a submersive system theory. Yet, in the moment of purpose/function
definition the system is static and therefore an immersive approach can be
taken. It should be noted that the terms purpose and function are very
distinct in Ackoffs terminology but they are not subject of this discussion
and therefore left for another time to evaluate.

\subsection{Transition- and transformation-pathways and problem solving} 

In this section the connection between WUMM's projects discussion on
transition and transformation pathways and Russell Ackoffs approach to problem
solving is elucidated. But first the seminar's title of co-operative planning
and action is brought into connection with the terms management and design as
we will use this terminology analog. In this context the term action can be
understood as active management in a dynamic context. On the other hand
planning can be understood as a design step much like envisioning, it has a
more static character.

Ackoff states in 1972 \cite{ackoff1994systems} that
\begin{quote}
  managers are educated to believe that a social systems's performance can be
  improved by improving the performance of each of its parts taken separately
  -- that is, if each part is managed well, the whole will be. This is seldom
  if ever the case, because parts that appear to be well managed when viewed
  separately seldom fit together well.
\end{quote} 

Ackoff describes four essential solutions (managements) to problems or messes.

\paragraph{Absolution}
ignores the problem with the expectation that it "solves itself" given enough
time.

\paragraph{Resolution}
can be seen as a quick fix. If focuses on clinical measures and is an approach
that results in a situation that is merely satisfactory. Its focus is on the
very specific problem rather than the general mechanism behind it.

\paragraph{Solution}
is within the given context the optimum. It is led by a research approach and
focuses on the general aspects of the problem.

\paragraph{Dissolution}
redesigns (planful) the entity or the environment where the problem arose.
This enables for a future state that is superior to the best possible in the
current one. It focuses on generality and uniqueness of the problem equally
and uses whatever technique seem to be fit.
\bigskip

The very concrete toolset that Ackoff provides stands against a more general
theoretical line of thought in the WUMM project.  Where it is less the
ambition to give concrete means of action, but rather to aggregate several
ways in wich change can happen. This is, in contrast to Ackoff, not even
determined to be governed by management in form of a managing entity, but can
also happen organically. As a starting point WUMM choose transition pathways
as proposed by \cite{geels2007typology}.

The WUMM project is concerned that the notions of transition and
transformation are not clearly distinguished in the systems literatur. From
other sciences they take that transition preserves system components whereas
transformation changes every componenent in a system
\cite[p.~78]{grabe2020seminar}.  As both notions are yet to be defined in the
WUMM ontology they are handled loosely and largely interchangeable here.
\begin{enumerate}
\item In production, if no landscape pressure, systems stays dynamically
  stable and reproduces itself.
\item In transformation, under moderate change in landscape, actors modify
  the direction of system development and innovation.
\item In de-alignment and re-alignment, under large sudden change in
  landscape, actors loose faith. De-alignment occurs, if no strong
  niche-innovation space exists, otherwise parallel niche-development compete,
  finally one niche-winner takes over in re-alignment to a new system.
\item Under technical substitution, large change in landscape occurs, a niche
  is developing, will break through and replace the current regime.
\item Reconfiguration appears, if no accute landscape pressure exists,
  niche-innovations are used to solve local problems, then the new solution
  spreads through the basic architecture of the system.
\item Disruptive change, large change in landscape, leads to a sequence of
  transitions and transformations and finally to a reconfiguration or possibly
  de- or re-alignment.
\end{enumerate}

In Ackoffs approaches one can determine some sort of quantitative structure.
If one was to apply a measure that indicates how ideal a solution to a problem
is in the most elaborate context (most detailed and comprehensive system view)
the ordering would most likely be in ascending order absolution, resolution,
solution and dissolution. In contrast, the extended transition pathways from
\cite{geels2007typology} are only qualitatively structured, as there is no
generally better or worse way.

Of course, Ackoffs ways of dealing with problems and messes can be optimal
within each given context. Imagine some problem on a cardiovascular machine
that needs to be fixed instantly before maybe redesigning the machine such
that the malfunction does not arise, but that holds for every thought
possible.  The question therefore arises, do, and if so, in which way Ackoffs
real world approaches relate to the transition pathways described and which
role does a submersive or immersive perspective play in this context?

In \cite[p.~77]{grabe2020seminar} the authors discuss some of the findings in
\cite{geels2007typology} and evaluate them against the seminar and especially
Holling's model of adaptive cycles in \cite{holling2001understanding}. A
similar approach is taken here, and Ackoffs ways of dealing with problems and
messes are mapped to Geels et al. transition pathways
\cite{geels2007typology}.

\paragraph{P0: In Production.}
This transition pathway can be linked to Ackoffs notion of absolution. There
is no external need for management as the system's inherent capabilities are
sufficient to adapt to a given situation.

\paragraph{P1: Transformation.}
This transition pathway relates to Ackoffs solution and resolution. There is a
change in system components that can be either local or global in the system,
but no redesign of the supersystem is taking place.

\paragraph{P2: De-Alignment and Re-Alignment.}
This transition pathway involves two main components: 
\begin{enumerate}[noitemsep]
\item The possibly of coexisting niche-innovations that can be understood as
  solutions.
\item A solution that dominates and thereby triggers possible dissolution.
\end{enumerate}

\paragraph{P3: Technical Substitution.}
This transition pathway can be seen as a local change and then matches best to
Ackoffs notion of a solution/resolution. It is a research driven change. It
can also be seen globally as a complete redesign and thereby would map to
dissolution. More specifically, if $S$ is the system under reference and its
component $S_i$ is substituted it would be a resolution or solution. But if
$S$ itself would be substituted or even changed to $S'$ it would be a
dissolution.

\paragraph{P4: Reconfiguration.}
This transition pathway is a combination of resolutions (local, niches) that
results in a sort of bubble up dynamics that can potentially lead to a
solution in the broader (basic) system. It might even extend to a dissolution
depending on the scope of basic change introduced. It could as well lead to
new problems and messes in the broader system.

\paragraph{P5: Disruptive change.}
This transition pathway is the most complex and branched pathway in Geels et
al. explainations due to its many conditionalities. There is no direct mapping
as it can potentially involve all of Ackoffs concepts of problem solving.  
\bigskip

It is appareant that the relation of these pairings highly depends on the
level from which one observes and is not very stringent.  Nevertheless, it
shows that linkages between those terminologies can be found. In any of the
aforementioned approaches two basic steps need to be taken. First, an entity,
and this can be one person, many persons or no person at all, is defining some
goal. This is happening either as a momentary snapshot or as a series of
those, e.g. a series of system states that are envisioned. This happens best
in a immersive/synthetic system context. As it is for the moment a vision of
where the system should be positioned in its super-system and what its I/O
properties should be. Then either the same entity or a different one needs to
actually implement this vision. This happens best in a submersive/synthetic
context as interaction relations and possible synergies need to be understood
to best deliver the given envisioned system state.

\section{Discussion}

In this section the findings from sections \ref{system_theory} and
\ref{system_view_and_management} are discussed.  While Ackoff intended to
develop a coherent terminology, WUMM tries to elucidate the relation between
different terminologies from different groups over the course of decades
\cite{grabe2020seminar}. Further Ackoffs approach is predominantly based on
his own thinking, without drawing connections to other authors. This is of
course probably due to the comparatively underpopular field of system sciences
in his time. Nevertheless we find that key concepts in system theory have
persisted and are shared between the WUMM system view and Ackoff. Especially
the connection between analysis/synthesis and submersion and immersion shows
that the aspect of perspective has an important role in system theoretical
thinking.

This in some sense triggers a switch of perspective from a quantitative to a
qualitative world view yielding tools such as Fuzzy Cognitive Mapping (FCM),
that finds a lot of use in the realm of participatory modeling
\cite{kosko1986fuzzy}. They are supposed to bridge the gap between shared
vision finding and implementation \cite{aguilar2005survey}.

\section{Conclusion}

To begin with it should be stated that complex systems are dynamic and can not
be analytically solved. Static systems can be solved given enough compute
power. Complex socio-techonological systems as discussed in WUMM can partially
be approximated using differential equations but likely not be solved in a
closed form. therefore optima are only local wich aligns with the projects
findings in \cite[p.~79]{grabe2020seminar}. further, only systems in motion
(dynamic) can be managed if one follows \cite{shchedrovitsky1981principles}.
so depending on the possibility if and where one draws the line between
management (action) and design (planning) one comes to the following
conclusions. if dynamic systems are submersive, then immersion is static.
immersion is comparable to ackoffs synthesis. ackoffs states that management
should not merely rely on analysis. Consequently, a mixture of submersive
(dynamic) and immersive (static) system views is needed, although we discuss
dynamic systems. This seems like a contradiction. But, in any given discrete
moment a vision of some system state $S$ is static although the system itself
remains dynamic. So one could argue, that a vision definition (design,
co-operative planning) should focus on synthesis and immersion whereas
management and implementation (action) should focus on analysis and
submersion.

This leads to the conclusion that any system theory that merely takes on one
of the aforementioned perspectives is insufficent for successful management in
the broader sense (planning and action).

\newpage

\bibliographystyle{abbrv}
\begin{thebibliography}{10}

\bibitem{shortspeech} Ackoff at an event to honoring the legacy of
  Dr. W. Edwards Deming.  \newblock
  \url{https://www.leancompetency.org/video/russell-ackoff-on-systems-thinking/}.
  \newblock 2021-05-05.

\bibitem{triz_system} The TRIZ Ontology Project:
  \url{https://github.com/wumm-project/RDFData/blob/master/Ontologies/GeneralConcepts.ttl}.

\bibitem{wumm_project} {The WUMM Project}.
  \url{https://wumm-project.github.io/}. 

\bibitem{ackoff1972note} R.~L. Ackoff.  \newblock A note on systems science.
  \newblock {\em Interfaces}, 2 (4):40-41, 1972.

\bibitem{ackoff1989data} R.~L. Ackoff.  \newblock From data to wisdom.
  \newblock {\em Journal of applied systems analysis}, 16 (1):3-9, 1989.

\bibitem{ackoff1994systems} R.~L. Ackoff.  \newblock Systems thinking and
  thinking systems.  \newblock {\em System Dynamics Review}, 10 (2-3):175-188,
  1994.

\bibitem{ackoff2006few} R.~L. Ackoff.  \newblock Why few organizations adopt
  systems thinking.  \newblock {\em Systems Research and Behavioral Science},
  23 (5):705, 2006.

\bibitem{aguilar2005survey} J.~Aguilar.  \newblock A survey about fuzzy
  cognitive maps papers.  \newblock {\em International journal of
    computational cognition}, 3 (2):27--33, 2005.

\bibitem{altshuller200240} G.~Altshuller.  \newblock {\em 40 principles: TRIZ
  keys to innovation}, volume~1.  \newblock Technical Innovation Center, Inc.,
  2002.

\bibitem{forrester1968industrial} J.~W. Forrester.  \newblock Industrial
  dynamics — after the first decade.  \newblock {\em Management Science},
  14 (7):398-415, 1968.

\bibitem{forrester1994system} J.~W. Forrester.  \newblock System dynamics,
  systems thinking, and soft or.  \newblock {\em System dynamics review},
  10 (2-3):245-256, 1994.

\bibitem{forrester1997industrial} J.~W. Forrester.  \newblock Industrial
  dynamics.  \newblock {\em Journal of the Operational Research Society},
  48 (10):1037-1041, 1997.

\bibitem{geels2007typology} F.~W. Geels and J.~Schot.  \newblock Typology of
  sociotechnical transition pathways.  \newblock {\em Research policy},
  36 (3):399-417, 2007.

\bibitem{grabe2020seminar} H.-G. Gr{\"a}be and K.~P. Kleemann.  \newblock {\em
  Seminar Systemtheorie: Universit{\"a}t Leipzig, Wintersemester 2019/20},
  Rohrbacher Manuskripte, volume~22.  \newblock Books on Demand, 2020.

\bibitem{hakuta1987synthesis} K.~Hakuta and L.~J. Gould.  \newblock Synthesis
  of research on bilingual education.  \newblock {\em Educational leadership},
  44 (6):38-45, 1987.

\bibitem{holling2001understanding} C.~S. Holling.  \newblock Understanding the
  complexity of economic, ecological, and social systems.  \newblock {\em
    Ecosystems}, 4 (5):390-405, 2001.

\bibitem{kosko1986fuzzy} B.~Kosko.  \newblock Fuzzy cognitive maps.  \newblock
  {\em International journal of man-machine studies}, 24 (1):65-75, 1986.

\bibitem{sengupta1965systems} S.~S. Sengupta and R.~L. Ackoff.  \newblock
  Systems theory from an operations research point of view.  \newblock {\em
    IEEE Transactions on Systems Science and Cybernetics}, 1 (1):9-13, 1965.

\bibitem{shchedrovitsky1981principles} G.~Shchedrovitsky.  \newblock
  Principles and general scheme of the methodological organization of system
  and structural research and development.  \newblock {\em Sistemnye
    issledovaniya. Metodologicheskie problemy: Ezhegodnik 1981}, p.  193-227,
  1981.

\bibitem{van2005new} E.~F. van Beeck, C.~M. Branche, D.~Szpilman,
  J.~H. Modell, and J.~J. Bierens.  \newblock A new definition of drowning:
  towards documentation and prevention of a global public health problem.
  \newblock {\em Bulletin of the World Health Organization}, 83:853-856,
  2005.

\bibitem{von1956general} L.~von~Bertalanffy.  \newblock General system theory.
  \newblock {\em General systems}, 1 (1):11-17, 1956.

\end{thebibliography}
\end{document}
