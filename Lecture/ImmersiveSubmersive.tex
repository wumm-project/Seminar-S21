\documentclass{beamer}
\usepackage{lsfolien}
\usepackage[english]{babel}
\usepackage[utf8]{inputenc}

\myfootline{System Modelling and Semantic Web -- Spring 2021}{Hans-Gert Gräbe}

\newcommand{\ueberschrift}[1]{\begin{center}\bf #1\end{center}}

\title{Modelling Sustainable Systems\\ and Semantic Web\\[6pt]
  Immersive and Submersive System Theories
  \vskip1em}

\subtitle{Lecture in the Module 10-202-2309\\ for Master Computer Science}

\author{Prof. Dr. Hans-Gert Gräbe\\
\url{http://www.informatik.uni-leipzig.de/~graebe}}

\date{May 2021}
\begin{document}

{\setbeamertemplate{footline}{}
\begin{frame}
  \titlepage
\end{frame}}

\section{Basics}
\begin{frame}{Relation Between two Systems in a Supersystem}
How can the relationship between two systems be conceptualised in their
environment?

According to our systems approach, only if we describe \emph{a larger system}
(supersystem) and the relation of the two subsystems to the supersystem as
\emph{components}.
\begin{center}  
  \begin{tikzpicture}[x=.4cm, y=.4cm, line width=1pt,
      myarrow/.style={blue, fill=green, -{Triangle[width = 18pt, length = 8pt]}, line width = 10pt}]
    \draw[fill=blue!20] (0.3,0) circle(2.2) ;
    \draw[fill=green]  (0,-.5) circle(.8) ;
    \draw[fill=red,opacity=.4] (0,.5) circle(.8);
    \draw[fill=blue!20] (12,0) circle(2.2);
    \draw[fill=green]  (7,-1.5) circle(.8) ;
    \draw[fill=red] (7,1.5) circle(.8);
    \draw[myarrow] (3.5,0) -- (5.5,0) ;
    \draw[->] (8,-1.4) -- node[below]{$f_2$} (9.7,-0.5) ;
    \draw[->] (8,1.4) -- node[above]{$f_1$} (9.7,0.5) ;
  \end{tikzpicture}
\end{center}
\end{frame}
\begin{frame}{Immersive Concept}

\ueberschrift{Mathematical formulation of the question}
\small
We look for functions $f_1: S_1 \rightarrow S$, $f_2: S_2 \rightarrow S$ with
certain properties.

For a "generic solution" we ask if for such a constellation exists a
\textbf{Universal Categorial Object}, i.e. a universal $U$ and universal maps
\begin{gather*}
  p_1: S_1 \rightarrow U,\ p_2: S_2 \rightarrow U,
\end{gather*}
such that for each triple $(f_1, f_2, S)$ the above constellation may be
written as
\begin{gather*}
  f_1 = f \circ p_1: S_1 \rightarrow U \rightarrow S,\ f_2 = f \circ p_2 : S_2
  \rightarrow U \rightarrow S
\end{gather*}
for a suitable $f = f_1 \oplus f_2: U \rightarrow S$.

\end{frame}
\begin{frame}{Immersive Concept}
$U$ is in some sense the "most general system" that combines the systems $S_1$
  and $S_2$ without "further effect".
  
$U$ in such a case is called a \textbf{direct sum} and we write $U = S_1
  \coprod S_2$.
\begin{center}
  \begin{tikzpicture}[
      myarrow/.style={blue, fill=green, -{Triangle[width = 18pt, length = 8pt]}, line width = 10pt}]
    \node at (-0.5,.5) (S1) {$S_1$} ; 
    \node at (-0.5,-.5) (S2) {$S_2$} ; 
    \node at (1,0) (S) {$S$} ;
    \draw[->] (S1) -- node[above]{$f_1$} (S) ; 
    \draw[->] (S2) -- node[below]{$f_2$} (S) ; 
    \node at (3.5,.5) (T1) {$S_1$} ; 
    \node at (3.5,-.5) (T2) {$S_2$} ; 
    \node at (5,0) (U) {$U$} ;
    \node at (6,0) (T) {$S$} ;
    \draw[myarrow] (1.8,0) -- (2.8,0) ; 
    \draw[->] (T1) -- node[above]{$p_1$} (U) ; 
    \draw[->] (T2) -- node[below]{$p_2$} (U) ; 
    \draw[->] (U) -- node[above]{$f$} (T) ; 
  \end{tikzpicture}
\end{center}
\end{frame}
\begin{frame}{Mathematical Categories}

Most mathematical models live in concrete \textbf{categories}, for example,
the category of sets, vector spaces, fibre bundles, algebraic varieties, and
so on.

Each such category is characterised by the fact that the terms \textbf{object}
and \textbf{morphism} have a clear meaning there.

Morphisms between vector spaces, for example, are operationally faithful
mappings, i.e. linear mappings that can be described by matrices for
finite-dimensional vector spaces.

Such universal objects do not exist in every category.

\emph{Remark:} The construction can easily be extended to finitely many $S_i$
and even to infinitely many $S_i, i\in I$, and so it is defined in
mathematics.
\end{frame}
\begin{frame}{Category of Sets}
In this category direct sums $U$ exist for both finite and infinite index sets
$I$. This is just the \textbf{disjunct union} of the sets $S_i$.

The maps $p_i$ are just the embeddings $p_i: S_i \rightarrow U$ of the partial
sets in their disjunct union.

The map $f: U \rightarrow S$ works as follows: For each $a\in U$ exists
exactly one $i$ and one $a'\in S_i$ with $a=p_i(a')$. Put $f(a)=f_i(a')$.

If $|S_1| = a, |S_2| = b$, then $|S_1 \coprod S_2| = a+b$.

The whole is no more than the sum of its parts.
\end{frame}

\begin{frame}{Submersive Concept}

\ueberschrift{Invert all Arrows (TRIZ Prinziple 13)}
\small
We look for functions $f_1: S_1 \leftarrow S$, $f_2: S_2 \leftarrow S$ with
certain properties.

Does for such a constellation exists a \textbf{Universal Categorial Object},
i.e. a universal $U$ and universal maps
\begin{gather*}
  p_1: S_1 \leftarrow U,\ p_2: S_2 \leftarrow U,
\end{gather*}
such that for each triple $(f_1, f_2, S)$ the above constellation may be
written as
\begin{gather*}
  f_1 = p_1 \circ f: S_1 \leftarrow U \leftarrow S,\ f_2 = p_2 \circ f : S_2
  \leftarrow U \leftarrow S
\end{gather*}
for a suitable $f = f_1 \otimes f_2: S \rightarrow U$. 

\end{frame}

\begin{frame}{Submersive Concept}
$U$ in such a case is called a \textbf{direct product} and we write $U = S_1
\prod S_2$.

\begin{center}  
  \begin{tikzpicture}[
      myarrow/.style={blue, fill=green, -{Triangle[width = 18pt, length = 8pt]}, line width = 10pt}]
    \node at (-0.5,.5) (S1) {$S_1$} ; 
    \node at (-0.5,-.5) (S2) {$S_2$} ; 
    \node at (1,0) (S) {$S$} ;
    \draw[<-] (S1) -- node[above]{$f_1$} (S) ; 
    \draw[<-] (S2) -- node[below]{$f_2$} (S) ; 
    \node at (3.5,.5) (T1) {$S_1$} ; 
    \node at (3.5,-.5) (T2) {$S_2$} ; 
    \node at (5,0) (U) {$U$} ;
    \node at (6,0) (T) {$S$} ;
    \draw[myarrow] (1.8,0) -- (2.8,0) ; 
    \draw[<-] (T1) -- node[above]{$p_1$} (U) ; 
    \draw[<-] (T2) -- node[below]{$p_2$} (U) ; 
    \draw[<-] (U) -- node[above]{$f$} (T) ; 
  \end{tikzpicture}
\end{center}

\end{frame}
\begin{frame}{Submersive Concept}
How does this change the perspective on the concept of system?

\begin{center}
  \begin{tikzpicture}[x=.4cm, y=.4cm, line width=1pt,
      myarrow/.style={blue, fill=green, -{Triangle[width = 18pt, length = 8pt]}, line width = 10pt}]
    \draw[fill=blue!20] (0.3,0) circle(2.2) ;
    \draw[fill=green]  (0,-.5) circle(.8) ;
    \draw[fill=red,opacity=.4] (0,.5) circle(.8);
    \draw[fill=blue!20] (12,0) circle(2.2);
    \draw[fill=green]  (7,-1.5) circle(.8) ;
    \draw[fill=red] (7,1.5) circle(.8);
    \draw[myarrow] (5.5,0) -- (3.5,0) ;
    \draw[<-] (8,-1.4) -- node[below]{$f_2$} (9.7,-0.5) ;
    \draw[<-] (8,1.4) -- node[above]{$f_1$} (9.7,0.5) ;
  \end{tikzpicture}
\end{center}
\end{frame}
\begin{frame}{Category of Sets}
In this category direct products $U$ exist for both finite and infinite index
sets $I$. This is just the \textbf{cartesian product} of the sets $S_i$.

The maps $p_i$ are just the projections $p_i: U \rightarrow S_i$ of the
product to the individual components.

The map $f: S \rightarrow U$ works as follows: For each $a\in S$ we set
$f(a)=(f_i(a))\in U$.

If $|S_1| = a, |S_2| = b$, then $|S_1 \prod S_2| = a \cdot b$.

\textbf{The whole is clearly more than the sum of its parts, most of the
  "information" is of relational nature.}
\end{frame}
\begin{frame}{Submersive and Immersive System Theories}
  
System theories rarely make a distinction between between these two
approaches.

To distinguish between the approaches, system theories in which the first
modelling principle dominates, are called \textbf{immersive system
  theories}. They can be recognised their constructions are essentially based
on embeddings (immersions).  

System theories that are based on the second modelling principle are called
\textbf{submersive system theories}.  They can be recognised by the fact that
their constructions are essentially based on projections (submersions) and
thus on processes of staggered complexity reduction.

The Theory of Dynamical Systems is a submersive system theory.
\end{frame}
\end{document}
